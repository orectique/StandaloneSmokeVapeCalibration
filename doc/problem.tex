\documentclass[]{article}

\usepackage[margin=3.4cm]{geometry}
\usepackage{amsmath}
\usepackage{amsfonts}
\setlength{\parindent}{0pt}
\setlength{\parskip}{0.2cm}
\renewcommand{\baselinestretch}{1}
\usepackage[obeyspaces]{url}
\usepackage{listings}

\usepackage[colorinlistoftodos,prependcaption,textsize=small]{todonotes}

% Default fixed font does not support bold face
\DeclareFixedFont{\ttb}{T1}{txtt}{bx}{n}{8} % for bold
\DeclareFixedFont{\ttm}{T1}{txtt}{m}{n}{8}  % for normal

% Tabular stretch
\def\arraystretch{1.5}

\usepackage{xltabular}
\usepackage{xcolor}
\newcommand\textr[1]{\textcolor{red}{#1}}

% Custom colors
\usepackage{color}
\definecolor{deepblue}{rgb}{0,0,0.5}
\definecolor{deepred}{rgb}{0.6,0,0}
\definecolor{deepgreen}{rgb}{0,0.5,0}
\usepackage{listings}
% Python style for highlighting
\newcommand\pythonstyle{\lstset{
		language=Python,
		basicstyle=\ttm,
		morekeywords={self},              % Add keywords here
		keywordstyle=\ttb\color{deepblue},
		emph={MyClass,__init__},          % Custom highlighting
		emphstyle=\ttb\color{deepred},    % Custom highlighting style
		stringstyle=\color{deepgreen},
		frame=tb,                         % Any extra options here
		showstringspaces=false
}}


% Python environment
\lstnewenvironment{python}[1][]
{
	\pythonstyle
	\lstset{#1}
}
{}

% Python for external files
\newcommand\pythonexternal[2][]{{
		\pythonstyle
		\lstinputlisting[#1]{#2}}}

% Python for inline
\newcommand\pythoninline[1]{{\pythonstyle\lstinline!#1!}}

% math mode text path-style formating
\newcommand\mt[1]{\ensuremath{\text{\path{#1}}}}

%opening
\title{Standalone Smoking Vaping Calibration}
\author{Tim Wilson}

\begin{document}
	
	\maketitle
	
	\section{Underlying Model}
	
	The underlying model for smoking and vaping is a compartmental flow model. The model tracks $220$ cohort, where a \emph{cohort} is a combination of age and sex. The model is initialised with \path{initialPrevlance.csv} runs from year 2021 to 2039 inclusive. Within each year, the model increments ages and then evaluates the flows in the model.
	
	\subsubsection{Increment Ages}
	The age of each cohort is incremented at the start of the year. In other words, the prevalence among a cohort of age $n$ at the start of year $t$ is equal to the prevalence of the cohort with age $n-1$ at the end of year $t-1$.
	
	Cohorts of age $\geq 110$ are removed, and cohorts of age $0$ are inserted each year. The newly inserted cohorts are initialised with the never smoker state ($n$) set to $1$.
	
	\subsubsection{Evaluate Flows}
	
	Flows are evaluated using a rough approximation, because the rates tend to be low. The quality of the optimisation depends on the number of time steps used.
	
	The flow rate are determined by reading the base rate from \path{flow_rate.csv} and applying an annual percentage change (APC), relative to the base year, from \path{flow_apc.csv}. The flow from state $x$ to state $y$ is denoted \path{x_y} in the file. Let $S$ be the states (compartments) of the model, then for $x,y \in S$, the flow rate in year $t$ is
	\begin{align*}
		r_{x, y, t} := b_{x,y} (1 + a_{x,y})^{t - 2021}
	\end{align*}
	where $b_{x, y}$ is base rate and $a_{x,y}$ is the APC.
	
	The flows are applied over a number of time steps, $s$, which defaults to $4$. Within each time step, the total flow out of a state, $x \in S$, is
	\begin{align*}
		f_{x, t} := \sum_{s \in S} r_{x, y, t}
	\end{align*}
	and the mass that flows from state $x$ to $y \in S$ is
	\begin{align*}
		p_{x, y, t} :=\left(1 - e^\frac{-f_{x, t}}{s}\right) \frac{r_{x, y, t}}{f_{x, t}}.
	\end{align*}
	Essentially, the flow \emph{rate} out of $x$ is being converted to a proportion, then split proportionally by the contribution each individual flow makes to the rate.
	
	The mass, $m_{x}$ of a state is then updated as follows
	\begin{align*}
		\text{out}_x :=& m_x\sum_{s \in S} p_{x,s} \\
		\text{in}_x :=& \sum_{s \in S}m_s p_{s,x} \\
		m_x \xrightarrow{\text{update}}& m_x - \text{out}_x + \text{in}_x
	\end{align*}
	
	This process is repeated $s$ times per year.
	
	\section{Optimisation}
	
	The model is 
	
\end{document}